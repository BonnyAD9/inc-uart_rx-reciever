\documentclass{article}

% Language setting
% Replace `english' with e.g. `spanish' to change the document language
\usepackage[czech]{babel}

% Set page size and margins
% Replace `letterpaper' with `a4paper' for UK/EU standard size
\usepackage[
    a4paper,top=2cm,bottom=2cm,left=2cm,right=2cm,marginparwidth=1.75cm
]{geometry}

% Useful packages
\usepackage{amsmath}
\usepackage{graphicx}
\usepackage[colorlinks=true, allcolors=blue]{hyperref}

\title{Návrh INC projektu UART}
\author{Jakub Antonín Štigler (xstigl00)}

\begin{document}
\maketitle

%\begin{abstract}
%Your abstract.
%\end{abstract}

\section{Architektura obvodu (RTL)}

\subsection{Schéma}
\includegraphics*[scale=0.4]{assets/RTL.png}

\subsection{Popis funkce}

Signál ze vstupu DIN je synchronizován pomocí tří D-KO v transparentním režimu.

Data jsou dále zavedena do demultiplexeru (DMX) který je rozvádí do
jednotlivých T-KO které tvoří výstup DOUT.

Obvod využívá 2 počítadla (CTR3 a CTR4) která praují na 3 a 4 bitech.
Když FSM výstup READ je 1 tak se počítadla zapojí za sebe a vytvoří jedno
sedmibitové počítadlo. Výstup z CTR3 určuje který bit ve výstupu se má
nastavovat. Přetečení z CTR4 do CTR3 je pomocí D-KO v transparentním režimu
zpožděno aby se bity zapisovaly na ještě starou pozici určenou CTR3.

Výstupní D-KO můžou být resetovány pomocí signálu (CLR) z FSM. Tento sygnál ale
musí být synchronizovaný z hodinami protože reset na D-KO je asynchronní.

\newpage

\section{Návrh automatu}

\subsection{Schéma automatu}

\subsubsection{Legenda}

\begin{itemize}
    \item Stavy: IDLE, OFFSET, READ, WAIT1, WAIT2, VALID
    \item Vstupní signály: DAT, CNT3, CNT4, CLK, RST
    \item Moorovy výstupy (a jejich implicitní hodnoty): READ=0, CC3=0, CC4=0,
          VLD=0, CLR=1
    \item Mealyho výstupy: žádné
\end{itemize}

\subsubsection{Schéma}

\includegraphics*[scale=0.9]{assets/FSM.pdf}

\subsection{Funkce}
Automat začíná ve stavu IDLE a čeká dokud se na vstupu DIN neoběví 0. Poté se
přesune do stavu OFFSET kde počká 8 cyklů hodin (přesun do prostřed bitu). Dále
se automat dostane do stavu READ kde vždy počká 16 cyklů hodin a přečte jeden
bit. Když takto přečte 8 bitů přesune se do stavu WAIT1 kde čeká dalších 16
cyklů hodin aby se dostal doprostřed ukončovacího bitu. Pak se přesune do stavu
WAIT2 a počká 8 cyklů hodin aby se dostal na konec ukončovacího bitu. Nakonec
se na jeden hodinový cyklus přesune do stavu VALID při kterém vyšle sygnál
DOUT\_VLD který říká že na DOUT je validní výstup.


%\bibliographystyle{alpha}
%\bibliography{sample}

\end{document}
