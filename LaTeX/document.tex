\documentclass{article}

% Language setting
% Replace `english' with e.g. `spanish' to change the document language
\usepackage[czech]{babel}
\usepackage{pdflscape}

% Set page size and margins
% Replace `letterpaper' with `a4paper' for UK/EU standard size
\usepackage[
    a4paper,top=2cm,bottom=2cm,left=2cm,right=2cm,marginparwidth=1.75cm
]{geometry}

% Useful packages
\usepackage{amsmath}
\usepackage{graphicx}
\usepackage[colorlinks=true, allcolors=blue]{hyperref}

\title{Návrh INC projektu UART}
\author{Jakub Antonín Štigler (xstigl00)}

\begin{document}
\maketitle

%\begin{abstract}
%Your abstract.
%\end{abstract}

\section{Architektura obvodu (RTL)}

\subsection{Schéma}
\includegraphics*[scale=0.4]{assets/RTL.png}

\subsection{Popis funkce}

Signál ze vstupu \emph{DIN} je synchronizován pomocí tří D-KO v transparentním
režimu.

Data jsou dále zavedena do demultiplexeru (DMX) který je rozvádí do
jednotlivých T-KO které tvoří výstup \emph{DOUT}.

Obvod využívá 2 počítadla (\emph{CTR3} a \emph{CTR4}) která praují na 3 a 4
bitech. Když \emph{FSM} výstup \emph{RD} je 1 tak se počítadla zapojí za sebe a
vytvoří jedno sedmibitové počítadlo. Výstup z \emph{CTR3} určuje který bit ve
výstupu se má nastavovat. Přetečení z \emph{CTR4} do \emph{CTR3} je pomocí D-KO
v transparentním režimu zpožděno aby se bity zapisovaly na ještě starou pozici
určenou \emph{CTR3}.

Výstupní D-KO můžou být resetovány pomocí signálu (\emph{CLR}) z FSM. Tento
signál ale musí být synchronizovaný z hodinami protože reset na D-KO je
asynchronní.

\newpage

\section{Návrh automatu}

\subsection{Schéma automatu}

\subsubsection{Legenda}

\begin{itemize}
    \item Stavy: \emph{Idle}, \emph{Offset}, \emph{Rds}, \emph{Wait1},
          \emph{Wait2}, \emph{Valid}
    \item Vstupní signály: \emph{DAT}, \emph{CNT3}, \emph{CNT4}, \emph{CLK},
          \emph{RST}
    \item Moorovy výstupy (a jejich implicitní hodnoty): \emph{RD}=0,
          \emph{CC3}=0, \emph{CC4}=0, \emph{VLD}=0, \emph{CLR}=1
    \item Mealyho výstupy: žádné
\end{itemize}

\subsubsection{Schéma}

\includegraphics*[scale=0.9]{assets/FSM.pdf}

\subsection{Funkce}
Automat začíná ve stavu \emph{Idle} a čeká dokud se na vstupu \emph{DIN}
neoběví 0. Poté se přesune do stavu \emph{Offset} kde počká 8 cyklů hodin
(přesun do prostřed bitu). Dále se automat dostane do stavu \emph{Rds} kde vždy
počká 16 cyklů hodin a přečte jeden bit. Když takto přečte 8 bitů přesune se do
stavu \emph{Wait1} kde čeká dalších 16 cyklů hodin aby se dostal doprostřed
ukončovacího bitu. Pak se přesune do stavu \emph{Wait2} a počká 8 cyklů hodin
aby se dostal na konec ukončovacího bitu. Nakonec se na jeden hodinový cyklus
přesune do stavu \emph{Valid} při kterém vyšle sygnál \emph{DOUT\_VLD} který
říká že na \emph{DOUT} je validní výstup.

\begin{landscape}
    \pagenumbering{gobble}
    \section{Simulace}
    \includegraphics*[scale=0.5751]{assets/simulation.png}
\end{landscape}

\end{document}
